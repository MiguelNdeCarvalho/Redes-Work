\documentclass[11pt]{article}   % tipo de documento e tamanho das letras

% os seguintes pacotes estendem a funcionalidade básica:
\usepackage[a4paper, total={16cm, 24cm}]{geometry} % tamanho da pagina e do texto
\usepackage[portuguese]{babel}  % traduz para portugues
\usepackage[utf8]{inputenc}
\usepackage{graphicx}           % graficos
\usepackage{amsmath}            % matematica
\usepackage{tikz}               % diagramas
    \usetikzlibrary{shadows}
\usepackage{booktabs}           % tabelas com  melhor aspecto
\usepackage[colorlinks=true]{hyperref}           % links para partes do documento ou para a web
\usepackage{listings}           % incluir codigo
    \renewcommand\lstlistingname{Listagem}  % Listing em portugues
    \lstset{numbers=left, numberstyle=\tiny, numbersep=5pt, basicstyle=\footnotesize\ttfamily, frame=tb,rulesepcolor=\color{gray}, breaklines=true}
\usepackage{blindtext}

% -------------------------------------------------------------------------------------------
\title
{
    \includegraphics[width=0.3\textwidth]{images/logo_universidade.png}
    \\[0.1cm]
    \textbf{Desenvolvimento de um Cliente e Servidor com um protocolo simples} \\
    Redes de Computadores
}

\author
{
    \textbf{Professores:} Luís Rato, Pedro Pathingo \\
    \textbf{Realizado por:} Miguel de Carvalho (43108), João Pereira (42864) 
}
\date{\today}

% -------------------------------------------------------------------------------------------
%                                Body                                                       %
% -------------------------------------------------------------------------------------------

\begin{document}
\maketitle

% -------------------------------------------------------------------------------------------
\section{Introdução} 

\hspace{0,5cm}Neste trabalho foi solicitado a realização de um programa que simule um \textbf{Chat} 
que seja composto por um \textbf{servidor} e por um ou vários \textbf{clientes}, utilizando um 
\textbf{protocolo de texto simples (plain text single line protocol)}.

O \textbf{servidor} deverá gerir vários clientes em \textbf{simultâneo} e ser responsável pela gestão:
\begin{itemize}
    \item da receção de mensagens;
    \item do envio de mensagens;
    \item dos diferentes canais, onde terá um canal "default" e outros canais alternativos;
    \item dos dados dos utilizadores.
\end{itemize}

O \textbf{cliente} apresenta dois modos de funcionamento:
\begin{itemize}
    \item anónimo, só consegue aceder ao \textbf{canal "default"};
    \item registado, consegue aceder a \textbf{todos os canais}.
\end{itemize}
\hspace{0,5cm}Em ambos os casos é \textbf{obrigatório} o uso de um \textbf{nickname}.

Os \textbf{clientes registados} podem apresentar 2 \textbf{roles (categoria especial)}:
\begin{itemize}
    \item cliente convencional;
    \item cliente operador, que permite gerir canais e gerir outros utilizadores.
\end{itemize}

O \textbf{protocolo de texto simples} é baseado em mensagens de texto, em que cada mensagem
tem um \textbf{comprimento máximo de 512 bytes} e é iniciada por um comando com 4 caracteres
e seguida de um ou mais parâmetros separados por espaços em branco.

% -------------------------------------------------------------------------------------------
\section{Implementação}

\hspace{0,5cm}Inicialmente pensámos como deveríamos proceder para realizar o trabalho, então começámos por 
definir como se encontram distribuídos os \textbf{512 Bytes} que correspondem ao tamanho
\textbf{máximo} da mensagem.

Posteriormente, certificamos-nos que o \textbf{Servidor} e o(s) \textbf{Cliente(s)} conseguiam
comunicar entre si, ou seja, que o protocolo estava a funcionar corretamente.

O próximo desafio foi a identificação dos \textbf{Clientes} que se encontram conectados ao servidor
e os \textbf{Clientes} que se encontram registados. Para solucionar isto, procedemos à criação de duas
\textbf{structs} onde esta informação iria ser guardada. Na \textbf{struct} \verb|ListaUser| é guardada
a informação dos \textbf{clientes} que estão \textbf{atualmente conectados ao servidor}.
Na \textbf{struct} \verb|ListaCliente| é guardada a informação de \textbf{todos os clientes que se encontram
registados}.

De seguida, começámos por proceder à implementação dos \textbf{comandos} tal como era pedido.

% -------------------------------------------------------------------------------------------
\section{Funções}
% -------------------------------------------------------------------------------------------
\section{Conclusão} % Conclusão
\hspace{0,5cm}Em suma, com a realização deste trabalho \textbf{"Chat"} ficámos muito 
mais esclarecidos sobre o seu funcionamento. \par
Saliento que nos ajudou a entender como funciona o \textbf{Potocolo de Texto Simples}, 
usado para proceder à comunicação entre o \textbf{Servidor} e o(s) \textbf{Cliente(s)}. 

% -------------------------------------------------------------------------------------------
\end{document}